\documentclass[12pt]{article}
\usepackage{fullpage,latexsym,amsthm,latexsym,amssymb}
\usepackage{amstext,amsfonts,amsmath,graphicx}
\usepackage{multicol}
\usepackage[usenames,dvipsnames]{color}
\usepackage{hyperref}


\oddsidemargin = -0.5 in
\addtolength{\textwidth}{0.8in}
\addtolength{\textheight}{0.2in}

%% Theorem statements %%
% THEOREMS -------------------------------------------------------
\newtheorem{theorem}{Theorem}[section]
\theoremstyle{definition}
\newtheorem{definition}{Definition}
\numberwithin{equation}{section}
%% MISC DEFINITIONS %%%%%%%%%%%%%%%%%%%%%%%%%%%%%%%%%%%

\newcommand{\ignore}[1]{}

\DeclareMathOperator{\shuffle}{shuffle}

\newcommand{\abi}[1]{\textcolor{Red}{#1}}
\newcommand{\qtrinfo}{CS181 Winter 2019}
\newcommand{\remove}[1]{}
\newcommand{\headnote}{
%\begin{multicols}{2}\small
\begin{itemize}
\item Please write your student ID \textbf{and the names of anyone you collaborated with} in the spaces provided and attach this sheet to the front of your solutions. \textbf{Please do not include your name anywhere since the homework will be blind graded.}
\item An extra credit of \textbf{5\%} will be granted to solutions written using \LaTeX. Here is one place where you can create latex documents for free: \url{https://www.overleaf.com/}. The link also has tutorials to get you started. There are several other editors you can use.
\item If you are writing solutions by hand, please write your answers in a neat and readable hand-writing.
\item Always explain your answers. When a proof is requested, you should provide a rigorous proof.
\item 20\% of the points will be given if your answer is ``I don't know''.
However, if instead of writing ``I don't know'' you write things that do not make any sense, no points will be given.
\item The homework is expected to take anywhere between 10 to 16 hours. You are
advised to start early.
\item Submit your homework online on the course webpage on Gradescope. 
\end{itemize}
%\end{multicols}
}

\newcommand{\hwhead}[2]{
    \raggedleft{Student ID: \underline{\hspace{1.2in}304990072\hspace{1.25in}} \\ \medskip
              Collaborators: \underline{\hspace{3in}} \\ \medskip}
  \bigskip
  \begin{center}
      {\LARGE{\qtrinfo \ -- Problem Set #1}}\\[0.3cm]
      {\Large{Due #2}}
  \end{center}
  \bigskip
  \raggedright
  \headnote
}

%%%%%%%%%%%%%%%%%%%%%%%%%%%%%%%%%%%%%%%%%%%%%%%%%%%%%
\addtolength{\topmargin}{-1cm}
\addtolength{\textheight}{2cm}






\begin{document}
\hwhead{1}{Tuesday, January 29, 11:59 pm}


\begin{center}
\fbox{%
   \parbox{0.8\linewidth}{
Note: \textit{All questions in the problem sets are challenging; you should not expect to know how to answer any question before trying to come up with innovative ideas and insights to tackle the question. If you want to do some practice problems before trying the questions on the problem set, \textbf{we suggest trying Exercise problems 1.4, 1.5, 1.9, 1.10, and 1.11 from the book. Do not turn in solutions to problems from the book.}}\\

\noindent The machines that we called ``Finite State Machines'' in class are also called "Deterministic Finite Automata (DFA)" and the machines we called
``Magical Finite State Machines'' in class are also called ``Non-Deterministic Finite Automata (NFA)''.
   }%
}
\end{center}


{\small \emph{Hint on all construction problems:  If you want to prove that $L$
is regular, it suffices to give an NFA for it. On the other hand, if you are
told to assume that $L'$ is regular, this means that there must exist a DFA
recognizing $L'$.} }

\newpage
\paragraph{Problem 1}
\subparagraph{} Assume L to be a regular language, $\exists M_1 = (Q_1, \Sigma, \delta_1, q_1, F_1)$ s.t. $L(M_1) = L$, and $M_1$ is a DFA. Define another NFA $M_2 = (Q_2, \Sigma, \delta_2, q_2, F_2)$ that accepts $L^R$ in the following way: \\
\begin{itemize}
  \item reverse all the arrows in $M_1$
  \item Add a new start state called $q_{new}$ which has arrows pointing to every end state of every accept states of $M_1$. All the arrows are labelled with $\epsilon$.
  \item Change the final states and the start state of $M_1$ into normal states.
\end{itemize}
More formally, define $M_2$ as follows: \\
Let $Q_2 = Q_1 \cup \{q_{new}\}$, $q_2 = q_{new}$, $F_2 = q_1$, $q_{new}$ is as defined above. $\delta_2$ is defind by if 
$p \in \delta_1(q,a)$, then $q \in \delta_1(p,a)$ \\~\\
Then $L(M_2) = L^R$ \\

$L^R \subseteq L(M_2)$: Consider $x \in L^R$. Then, by definition of $L^R$, $x^R \in L$. Thus, on input $x_R$, $M_1$ goes from the starting state $q_1$ to the a state in $F_1$. Since $M_2$ is defined by reversing all transitions in $M_1$, and adding a new state $q_{new}$ with $\epsilon$-transitions to states in $F_1$, an input x to $M_2$ will be accepted with a reversed sequence of transitions of states (with an additional start state of $q_{new}$). Hence, $L^R \subseteq L(M_2)$. \\~\\

$L(M_2) \subseteq L^R$: Consider $y \in L(M_2)$. Then, on input y, $M_2$ goes from state $q_{new}$ to a state $q' \in F_2$. Since $q_{new}$ is not in $F_2$,then $M_2$ must first transition out of $q_{new}$. But, the only transitions from $q_{new}$ are the $\epsilon$-transitions to states in $F_1$. From this point onwards, every transition of states $M_2$ takes have a corresponding transition in $M_1$, but in a different direction. Also, $M_2$ has only 1 final state, which is the start state of $M_1$. Hence, for every $y \in L(M_2)$ , $y^R \in M_1$. $\Rightarrow y \in L^R$ by definition. Hence, $L(M_2) \subseteq L^R$. \\~\\

Therefore, $L(M_2) = L^R$, $L^R$ is a regular language. (proven)

\newpage
\paragraph{Problem 2}
\subparagraph{} Assume L to be a regular language,  $\exists M_1 = (Q_1, \Sigma, \delta_1, q_1, F_1)$ s.t. $L(M_1) = L$, and $M_1$ is a DFA. Define another NFA $M_2 = (Q_2, \Sigma, \delta_2, q_2, F_2)$ that accepts $L_{\frac{1}{2}-}$ in the following way: \\~\\

The idea is, if a string of length $l$ is accpeted by $M_2$, then at least a string of length $2l$ can be accepted by $M_1$. Since the length $l$ is a variable, we need keep track of both the current state and if it is possible to reach a state $\in F_1$ in exactly $l$ steps. More formally: \\~\\


Let $M_2=(Q_2, \Sigma, \delta_2, q_2, F_2)$ be an NFA such that\\
\begin{itemize}
	\item $Q_2 = (Q_1\times Q_1)\cup \{q_2\}$, the first entry denotes the current state in $M_1$ if the string is an input for $M_1$, the second entry denotes the states that are possible to reach a state $\in F_1$ in $l$ steps.\\
	\item $q_2$ is the start state we define for $M_2$\\
	\item $F_2$ are states with format $(q,q|q\in Q_1)$ according to our definition of $Q_2$. This is because if the string is used as an input for $M_1$, we can reach a final state $\in F_1$ in $l$ steps ($l$ is the current length of string).
	\item $\delta_2(q, a)=
    \begin{cases}
               (q_1, f) \textbf{ for }f \in F_1 &\textbf{if}\ q = q_2,\ a = \epsilon\\
               (\delta_1(q,a),m) &\textbf{if}\ q = (x,y)\textbf{, } x,y \in (Q_1\times Q_1),\ a \neq \epsilon,y=\delta_1(m,a)\\
    \end{cases}$\\~\\~\\
\end{itemize}    
For $q \in Q_2, a \in \Sigma \cup \{\epsilon\}.$\\
We define $\delta_2$ in such a way because each time we get a new input, $l$  increases by 1. Thus we need to change the second entry to a state in $Q_1$ that is 1 more step further from the final states of $M_1$. The only exception is when $M_2$ is in the start state, we need to indicate that an empty string is given to $M_1$, $M_1$ is at its start state $q_1$, and all the final states of $M_1$ are 0 steps away from themselves (due to a string length $l$ of 0). \\~\\

$L_{\frac{1}{2}-} \in L(M_2)$: according to out construction of $M_2$, all string $x \in L_{\frac{1}{2}-}$ are accepted by $M_2$ \\~\\
$L(M_2) \in L_{\frac{1}{2}-}$: all the final states of $M_2$ has the form $(q,q|q\in Q_1)$ according to our definition. This means if the same string is used as input for $M_1$, the $M_1$ will end in a state which can reach one of its final states in $l$ steps, $l$ is the length of the input string. Hence, according to the definition of $L_{\frac{1}{2}-}$, $L(M_2) \in L_{\frac{1}{2}-}$. \\~\\

Therefore, $L(M_2) = L_{\frac{1}{2}-}$, $L_{\frac{1}{2}-}$ is a regular language. (proven)

\newpage
\paragraph{Problem 3}
\subparagraph{} Assume there exists a DFA $M_1 = (Q_1, \Sigma_1, \delta_1, q_1, F_1),\ \Sigma_1 = \Sigma\cup\{\$\}$ such that it accepts a string in the form of $x\$x$. The object is to prove $L(M_{2p}) = \{x \in \Sigma^*\ |\ M_1\ accepts\ x\$x\}$ is a regular language. To do this, define NFA $M_2 = (Q_2, \Sigma, \delta_2, q_2, F_2)$ by modifying $M_1$. We can use the same set of states as $M_1$, and add a new start state $q_2$ with transitions labelled with $\epsilon$ to every state q that accepts a transition labelled with \$. Then, delete all transitions in the original $M_1$ labelled with \$. more formally: \\
\begin{itemize}
	\item $Q_2 = Q_1 \cup \{q_2\}$\\
	\item $q_2$ is the new start state we defined
	\item $F_2 = F_1$ \\
	\item $\Sigma_3 = \Sigma \cup \{\epsilon\}$ \\
	\item For $q \in Q_2, a\ \in \Sigma,$ \\
	$\delta_2(q, a)=
    \begin{cases}
               \delta_1(q_1, a) &\textbf{if}\ q \neq q_2,\ a \in \Sigma \\
               \{q\ |\ q = \delta_1(q^*, \$),\ q^* \in Q\},  &\textbf{if}\ q = q_2,\ a = \epsilon\\
    \end{cases}$\\~\\~\\
\end{itemize}    

Since every NFA has an equivalent DFA, we can construct a DFA $M_3$ equivalent to $M_2$, with $\Sigma_3 = \Sigma$. \\~\\

$L(M_{2p}) \subseteq L(M_2) \Rightarrow L(M_{2p}) \subseteq L(M_3)$: For any input in the form of $x\$x$ (accepts by $M_{2p}/M_1$),  $M_1$ will be in a state that accepts a transition labelled with \$ after reading the \$ symbol in $x\$x$. If the same $x$ is used as an input to $M_2$, since the only transitions $q_2$ has are those $\epsilon-transitions$ to $\{q\ |\ q = \delta_1(q^*, \$),\ q^* \in Q\}$, $M_2$ can be in the same state when $M_1$ reads \$. Since $M_1$ accepts $x\$x$, $M_1$ will continue from that state until it reaches an accept state after reading $x\$x$. The same set of transitions will take place in $M_2$ since $\delta_2(q, a)$ is defined as $\delta_1(q_1, a)$ for $q \neq q_2$, and $M_2$ will never go back to the start state $q_2$ after the $\epsilon=transitions$ as defined. Thus for any $x$ that is accepted by $M_{2p}$, it is accepted by $M_2$. Hence, $L(M_{2p}) \subseteq L(M_2)$. $\Rightarrow  L(M_{2p}) \subseteq L(M_3)$ \\~\\

$L(M_2)\subseteq L(M_{2p}) \Rightarrow L(M_3)\subseteq L(M_{2p})$: Assume not. For any string $x$ accepted by $M_2$, according to the definition of $\delta_2$ (same as $\delta_1$ except when $q = q_0$), there must be a string in the form of $y\$x$ that is accepted by $M_1$. Thus, if $L(M_2)\nsubseteq L(M_{2p})$, $y \neq x$. Since $M_1$ only accepts strings in the form of $x\$x$, $M_1$ does not accept the string $y\$x$ if $y \neq x \Rightarrow$ contradiction. Hence, $L(M_2)\subseteq L(M_{2p}) \Rightarrow$ $L(M_3)\subseteq L(M_{2p})$. \\~\\

Hence, $L(M_3)= L(M_{2p})$, $L(M_{2p})$ is a regular language. (proven)


\end{document}